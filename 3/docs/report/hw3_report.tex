%
%
\documentclass[12pt,twoside]{article}

\input{macros}

\usepackage{amsmath}
\usepackage{url}
\usepackage{mdwlist}
\usepackage{graphicx}
\usepackage{clrscode3e}
\newcommand{\isnotequal}{\mathrel{\scalebox{0.8}[1]{!}\hspace*{1pt}\scalebox{0.8}[1]{=}}}
\usepackage{listings}
\usepackage{tikz}
\usepackage{float}
\usetikzlibrary{arrows}
\usetikzlibrary{matrix}
\usetikzlibrary{positioning}
\usetikzlibrary{shapes.geometric}
\usetikzlibrary{shapes.misc}
\usetikzlibrary{trees}

\usepackage{hyperref}
\usepackage[all]{hypcap}
\usepackage{caption}
\usepackage{subfigure}
\captionsetup{hypcap=true}

\newcommand{\answer}{
 \par\medskip
 \textbf{Answer:}
}

\newcommand{\collaborators}{ \textbf{Collaborators:}
%%% COLLABORATORS START %%%

\tabT Name: Zhuo Chen

\tabT Student ID: 3170101214
%%% COLLABORATORS END %%%
}

\newcommand{\answerIa}{ \answer
%%% PROBLEM 1(a) ANSWER START %%%
The test error is 92.7\%.
%%% PROBLEM 1(a) ANSWER END %%%
}

\newcommand{\answerIIa}{ \answer
%%% PROBLEM 2(a) ANSWER START %%%
\begin{figure}[h]
	\centering
	\subfigure{
		\includegraphics[width=.33\linewidth]{imgs/knn_1}
	}\subfigure{
		\includegraphics[width=.33\linewidth]{imgs/knn_10}
	}\subfigure{
		\includegraphics[width=.33\linewidth]{imgs/knn_100}
	}
	\caption{KNN boundary with different K}
\end{figure}
%%% PROBLEM 2(a) ANSWER END %%%
}


\newcommand{\answerIIb}{ \answer
%%% PROBLEM 2(b) ANSWER START %%%
Cross validation.
%%% PROBLEM 2(b) ANSWER END %%%
}

\newcommand{\answerIIc}{ \answer
	%%% PROBLEM 2(c) ANSWER START %%%
\begin{figure}[h]
	\centering
	\includegraphics[width=.5\linewidth]{imgs/knn_hack}
	\caption{KNN-based CAPTCHA recognization results}
\end{figure}
	%%% PROBLEM 2(c) ANSWER END %%%
}



\newcommand{\answerIIIa}{ \answer 
%%% PROBLEM 3(a) ANSWER START %%%
\begin{figure}[h]
	\centering
	\includegraphics[width=.9\linewidth]{imgs/id3}
	\caption{The decision tree}
\end{figure}
%%% PROBLEM 3(a) ANSWER END %%%

}

\newcommand{\answerIIIIa}{ \answer
	%%% PROBLEM 4(a) ANSWER START %%%
\begin{figure}[h]
	\centering
	\subfigure{
		\includegraphics[width=.35\linewidth]{imgs/kmeans_min}
	}\subfigure{
		\includegraphics[width=.35\linewidth]{imgs/kmeans_max}
	}
	\caption{Trials with min(left) and max(right) SD}
\end{figure}

	%%% PROBLEM 4(a) ANSWER END %%%
}

\newcommand{\answerIIIIb}{ \answer
	%%% PROBLEM 4(b) ANSWER START %%%
	When we initialize the centriods, we can let the first random, and then select the sample with the largest distance to selected centroids as the next centroid.
	
	%%% PROBLEM 4(b) ANSWER END %%%
}

\newcommand{\answerIIIIc}{ \answer
	%%% PROBLEM 4(c) ANSWER START %%%
\begin{figure}[h]
	\centering
	\subfigure{
		\includegraphics[width=.5\linewidth]{imgs/kmeans_10}
	}\vspace{-3ex}
	\subfigure{
		\includegraphics[width=.5\linewidth]{imgs/kmeans_20}
	}\vspace{-3ex}
	\subfigure{
		\includegraphics[width=.5\linewidth]{imgs/kmeans_50}
	}
	\caption{Trials with min(left) and max(right) SD}
\end{figure}
	%%% PROBLEM 4(c) ANSWER END %%%
}

\newcommand{\answerIIIId}{ \answer
	%%% PROBLEM 4(d) ANSWER START %%%
	25\%.
\begin{figure}[h]
	\centering
	\subfigure{
		\includegraphics[width=.45\linewidth]{imgs/vq}
	}\vspace{-5ex}
	\subfigure{
		\includegraphics[width=.5\linewidth]{imgs/vq_8}
	}\subfigure{
		\includegraphics[width=.5\linewidth]{imgs/vq_16}
	}\vspace{-5ex}
	\subfigure{
		\includegraphics[width=.5\linewidth]{imgs/vq_32}
	}\subfigure{
		\includegraphics[width=.5\linewidth]{imgs/vq_64}
	}
	\caption{Compressed images}
\end{figure}
	%%% PROBLEM 4(d) ANSWER END %%%
}

\setlength{\oddsidemargin}{0pt}
\setlength{\evensidemargin}{0pt}
\setlength{\textwidth}{6.5in}
\setlength{\topmargin}{0in}
\setlength{\textheight}{8.5in}

% Fill these in!
\newcommand{\theproblemsetnum}{3}
\newcommand{\releasedate}{May 21, 2020}
\newcommand{\partaduedate}{Monday, June 8}
\newcommand{\tabUnit}{3ex}
\newcommand{\tabT}{\hspace*{\tabUnit}}

\begin{document}

\handout{Homework \theproblemsetnum}{\releasedate}

%\textbf{Both theory and programming questions} are due {\bf \partaduedate} at
%{\bf 11:59PM}.
%
\collaborators
%Please download the .zip archive for this problem set.
% Your grade will be based on both your solutions and the grading explanation.

\medskip

\hrulefill

\begin{problems}

\problem \textbf{Neural Networks}

In this problem, we will implement the feedforward and backpropagation process of
the neural networks.
\begin{problemparts}
\problempart 
\answerIa

\end{problemparts}

\problem \textbf{K-Nearest Neighbor}

In this problem, we will play with K-Nearest Neighbor (KNN) algorithm and try it on
real-world data. Implement KNN algorithm (in \emph{knn.m}/\emph{knn.py}), then answer the following questions.
\begin{problemparts}
\problempart
Try KNN with different K and plot the decision boundary.


\answerIIa

\problempart We have seen the effects of different choices of K. How can you choose a proper K
when dealing with real-world data ?

\answerIIb

\problempart Finish \emph{hack.m}/\emph{hack.py} to recognize the CAPTCHA image using KNN algorithm.

\answerIIc
\end{problemparts}

\problem \textbf{Decision Tree and ID3}

Consider the scholarship evaluation problem: selecting scholarship recipients based on gender
and GPA. Given the following training data:

\answerIIIa


\problem \textbf{K-Means Clustering}

Finally, we will run our first unsupervised algorithm – k-means clustering.
\begin{problemparts}
	\problempart
	Visualize the process of k-means algorithm for the two trials.
	
	\answerIIIIa
	
	\problempart How can we get a stable result using k-means?
	
	\answerIIIIb
	
	\problempart  Visualize the centroids.
	
	\answerIIIIc
	
	\problempart  Vector quantization.
	
	\answerIIIId
	
\end{problemparts}


\end{problems}
\end{document}
